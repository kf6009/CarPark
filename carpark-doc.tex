%% Dr Alun Moon
%% assignment-template.tex, generated from assignment-template.dtx
\documentclass[12pt]{article}
\usepackage[british]{babel}

\usepackage{url}
\usepackage[round]{natbib}
\usepackage{infoboxes}

\usepackage{latexsym}
\usepackage{tlatex}
\usepackage{color}
\definecolor{boxshade}{gray}{.8}
\setboolean{shading}{true}

\usepackage{enumerate,
            siunitx}

\begin{document}

\section{Car-park Specification}\label{lms.report}
\subsection{The Specification} 
\begin{tla}
------------------------------ MODULE carpark ------------------------------
EXTENDS Naturals, FiniteSets
CONSTANT Capacity, Cars
VARIABLE carpark

TypeInvariant == carpark \in SUBSET Cars   \* Car park contains some set of cars

Safety == Cardinality(carpark) \leq Capacity \* Car park is not over capacity 
 

Init == carpark = {}  \* an empty carpark
    
Enters(car) == 
    /\ Cardinality(carpark) < Capacity \* is there space ?
    /\ car \notin carpark  \* a car in the carpark cant enter 
    /\ carpark' = carpark \union {car} \* A car enters the carpark

Leaves(car) ==
    /\ car \in carpark \* only cars in the carpark can leave
    /\ carpark' = carpark \ {car}
    
Next ==
    \/ \E car \in Cars : Enters(car)
    \/ \E car \in carpark : Leaves(car)

Spec == Init /\ [][Next]_carpark
=============================================================================
\end{tla}
\begin{tlatex}
\@x{}\moduleLeftDash\@xx{ {\MODULE} carpark}\moduleRightDash\@xx{}%
\@x{ {\EXTENDS} Naturals ,\, FiniteSets}%
\@x{ {\CONSTANT} Capacity ,\, Cars}%
\@x{ {\VARIABLE} carpark}%
\@pvspace{8.0pt}%
\@x{ TypeInvariant \.{\defeq} carpark \.{\in} {\SUBSET} Cars\@s{8.2}}%
\@y{%
  Car park contains some set of cars
}%
\@xx{}%
\@pvspace{8.0pt}%
\@x{ Safety \.{\defeq} Cardinality ( carpark ) \.{\leq} Capacity}%
\@y{%
  Car park is not over capacity 
}%
\@xx{}%
\@pvspace{16.0pt}%
\@x{ Init \.{\defeq} carpark \.{=} \{ \}\@s{4.1}}%
\@y{%
  an empty carpark
}%
\@xx{}%
\@pvspace{8.0pt}%
\@x{ Enters ( car ) \.{\defeq}}%
\@x{\@s{16.4} \.{\land} Cardinality ( carpark ) \.{<} Capacity}%
\@y{%
  is there space ?
}%
\@xx{}%
\@x{\@s{16.4} \.{\land} car \.{\notin} carpark\@s{4.1}}%
\@y{%
  a car in the carpark cant enter 
}%
\@xx{}%
\@x{\@s{16.4} \.{\land} carpark \.{'} \.{=} carpark \.{\cup} \{ car \}}%
\@y{%
  A car enters the carpark
}%
\@xx{}%
\@pvspace{8.0pt}%
\@x{ Leaves ( car ) \.{\defeq}}%
\@x{\@s{16.4} \.{\land} car \.{\in} carpark}%
\@y{%
  only cars in the carpark can leave
}%
\@xx{}%
 \@x{\@s{16.4} \.{\land} carpark \.{'} \.{=} carpark \.{\,\backslash\,} \{ car
 \}}%
\@pvspace{8.0pt}%
\@x{ Next \.{\defeq}}%
\@x{\@s{16.4} \.{\lor} \E\, car \.{\in} Cars \.{:} Enters ( car )}%
\@x{\@s{16.4} \.{\lor} \E\, car \.{\in} carpark \.{:} Leaves ( car )}%
\@pvspace{8.0pt}%
\@x{}\bottombar\@xx{}%
\end{tlatex}

\section{The Model}
The car-park system has been run with two models, of different scales.

\subsection{Model 1 Overview}
Model 1 is a small model to quickly prove correctness of the specification.
\paragraph{The Behaviour specification} is an \emph{initial
predicate and next-state relation}.  These are filled in automatically if
there is an \textit{Init} and \textit{Next} predicates. 
\paragraph{The Model} values assigned to declared constants
\subparagraph{The set of cars to use} is $\mathit{Cars}\leftarrow 1..10$
\subparagraph{The capacity of the car-park} is set
$\mathit{Capacity}\leftarrow 5$
\subsection{Checks and verifications}
\paragraph{Invariants} Two invariants are checked
\subparagraph{The type invariant} \textit{TypeInvariant} 
\subparagraph{The safety case} the \textit{Safety} predicate.

\subsection{Results} A summary of the results
\paragraph{Statistics} a summaries of the actions and number of states
found.

\begin{table}[h]
\begin{tabular}{lr}
	States found & \num{5121} \\
 Distinct states & \num{638} \\ 
\end{tabular}
\end{table}

\begin{table}[h]
\begin{tabular}{llrr}
	\textbf{Action} & Location & States Found & \textbf{Distinct states} \\
	\hline
	\textit{Init}   & Line 11 & 1 & 1 \\
	\textit{Enters} & Line 13 & \num{2560} & 637 \\
	\textit{Next}   & Line 24 & \num{2560} & 0 
\end{tabular}
\end{table}

\subsection{Model 2 Overview}
Model 2 is a large model to show how quickly states can build up.
Apart from the model parameters, all the tests are the same.

\paragraph{The Model} values assigned to declared constants
\subparagraph{The set of cars to use} is $\mathit{Cars}\leftarrow 1..100$
\subparagraph{The capacity of the car-park} is set
$\mathit{Capacity}\leftarrow 10$

\subsection{Results} A summary of the results
\paragraph{Statistics} a summaries of the actions and number of states
found.  TLC (the model checker) took 27 minutes to find the number of states
shown.  After a further 20-hours of runtime the model-checking had not
completed and was terminated.  The remaining queue-size shows that there are
still large numbers of states to add to the statistics.

\begin{table}[h]
\begin{tabular}{lr}
	States found & \num{1375477467} \\
	Distinct states & \num{285379380} \\ 
	Queue size 		& \num{271624604} \\
\end{tabular}
\end{table}

\begin{table}[h]
\begin{tabular}{llrr}
	\textbf{Action} & Location & States Found & \textbf{Distinct states} \\
	\hline
	\textit{Init} & Line 11 & 1 & 1 \\
	\textit{Enters} & Line 13 & \num{1310963481} & \num{284444823} \\
	\textit{Next} & Line 24 & \num{64513985} & \num{934556} 
\end{tabular}
\end{table}

\subsection{Discussion}
\subsubsection{Model description} 
\paragraph{The state of the car-park is the set of cars it contains.}
For the purposes of
the specification we are not concerned with how the cars are arranged in the
car-park or the order in which they enter and leave.  

So a \emph{set} is appropriate to model the state.  Using a finite-set allows
us to count the size of the set; how many cars are in the car-park.

\subparagraph{The initial conditions} are that the car-park is empty.\\
(The empty set \{\} )

\subparagraph{The type invariant} is that the set modelling the car-park is
some subset of the set of Cars in the model.

\subparagraph{The safety invariant} Is that the size of the set of cars in the
car-park should not exceed the capacity.  As the state-variable is a
finite-set, it could be argued that this is part of the type-invariant.

\paragraph{The enter operation} has a guard condition that
there are spaces available, and that a car is not already in the car-park.
The state is updated by adding the car to the set of cars in the car-park.

\paragraph{The exit operation} is guarded by the condition
that a car must be in the car-park to exit.  The state is updated removing
the car from the car-park

\paragraph{The Next relation} is that either there is a car that can enter the
car-park, or that there is a car in the car-park that can exit.

\subsubsection{Interpretation of results}
The specification makes no assumptions about the distribution of cars in the
car-park, nor the order they enter and leave.

Model 1 is small enough to run in a reasonable amount of time, and is
sufficient to verify the specification.

Model 2 generates a huge number of states and consequently requires a long
time to run.  The results from it add nothing to the verification of the
model.  

\begin{infobox}{\warning}
	Be careful not to use an over-large set of model data.  Larger models
	often add little extra information.  The number of states and runtime of
	the model checking  grow exponentially.

	\paragraph{The moral of the tale} use as small a model as is practicable
	to fully check the specification.
\end{infobox}

\end{document}

\endinput
%%
%% End of file `assignment-template.tex'.
